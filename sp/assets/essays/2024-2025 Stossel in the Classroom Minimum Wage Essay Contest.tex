\documentclass[12pt]{article}
\usepackage[utf8]{inputenc}
\usepackage{geometry}
\usepackage{hyperref}
\usepackage{setspace}
\geometry{margin=1in}
\setstretch{1.2}
\setlength{\parskip}{1em}


\title{2024-2025 Stossel in the Classroom Minimum Wage Essay Contest}
\author{Kalie (Kangli) Jia}
\date{April 27, 2025}

\begin{document}

\maketitle

Minimum wage is a widely debated topic that affects employers and employees across numerous sectors. Increasing the minimum wage is supposed to allow low-paid workers to keep up with rising costs of living. However, having to pay workers a higher wage may cause employers to lay off workers and raise their products' prices. At most, minimum wage only provides a temporary method for the working class to live. Instead of, "should we increase the minimum wage?", we should ask, "what other methods can solve the problems that minimum wage is trying to solve?".

Some countries, including Denmark, Finland, and Sweden, don't have a minimum wage law. Instead, livable wages are determined through sectoral collective bargaining. Together, employers and worker unions negotiate working standards that best suit both sides' needs. This system allowed McDonald's to pay its Danish employees \$22/hr, with six-week vacations and paid maternity leaves [1]. In July 2024, a Big Mac costed \$5.66 in Denmark while costing \$5.69 in the U.S [2]. This comparison suggests that improving working conditions doesn't directly cause products to be more expensive. Municipalities in the U.S. also follow a collective bargaining system to determine work contracts. I interviewed the HR Coordinator of Bedford Public Schools in Massachusetts. They explained, "Through collective bargaining, pay scales are based on factors like degree level, years of experience, and the budget the town invests towards education." Unprotected workers under large corporations can unionize. The voices of workers will be heard when they speak up as a group.

As of 2012, 66\% of workers earning the federal minimum wage or less were employed in corporations with over 100 employees, mainly in retail and food services sectors [3]. I interviewed Market Basket in Billerica, MA. They stated, "We pay almost double the federal minimum wage in all of our establishments across New Hampshire. Our employees work their way up to managerial positions over time." Employees can only develop their skills if they have enough working hours. Higher wages decrease employees' working opportunities, hindering their chances of promotion.

Teenagers commonly enter the workforce with a minimum wage job. 45\% of workers earning the federal minimum wage are under the age of 25 as of 2022 [4]. Town programs offer seasonal jobs, where the starting pay varies around the state minimum wage. At my interview, the HR Director of Bedford, MA explained, "Even though municipalities are exempt from the state's minimum wage requirements, most of these positions offer more than minimum wage in order to encourage people to apply." Unlike the adults that depend on their wages for a living, most youth value these jobs as opportunities to prepare for their futures. Too high of a minimum wage decreases the number of opportunities available for them.

Nevertheless, certain industries like the restaurant industry do not follow standard minimum wage laws. Most restaurants are small businesses whose owners struggle to provide a base pay for their employees—not to mention dealing with unions. Instead, small restaurants are supported by customers' tips. Tipping culture allowed the American restaurant industry to grow to its uniquely large size, which reached \$782 billion in 2016 [5]. Due to its complexity, removing tipping culture or raising the minimum wage may hurt the American restaurant industry more than help it.

The U.S. government can help by improving the current system of determining the minimum wage. Right now, Congress is responsible for determining the rate. Unfortunately, Congress members are not experts who thoroughly understand individual industries. Furthermore, the political and partisan nature of the U.S. system makes the decision-making process slow. This inefficiency prevents the minimum wage from being updated at a steady frequency. In 2007, the federal minimum wage leaped from \$5.15/hr to \$7.25/hr. This sudden increase of 41\% shocked businesses, especially small business owners. Today, the minimum wage has been sitting at \$7.25 for 18 years, while inflation has grown by about 55\% [6]. If the federal government raised the minimum wage to \$15/hr, it would mean a sudden 107\% increase and unprepared businesses will struggle to stay open.

By contrast, the E.U. adopted a directive in 2022 to maintain adequate minimum wages. The E.U. council set a framework to determine the minimum wage, and each member state applies the framework onto their own circumstances. For member states that already have a national minimum wage, the directive requires it to be at least 60\% of the country's median wage or 50\% of its average wage [7]. The directive also states that minimum wages need to be updated at least biannually [8]. Updating the minimum wage frequently maintains its adequacy and gives time for businesses to adjust. The U.S. federal government can also define a framework for setting the minimum wage and each state can apply it onto their own circumstances.

Moreover, the U.S. government can help both employers and employees by taxing less. Major tax-funded programs—like healthcare, insurance, and infrastructure—can be reformed into more cost-efficient systems. This will maximize the societal benefit that these programs output per societal cost. With less taxes, businesses will flourish, and workers can have more money to spend. As a result, the economy will circulate faster.

Even though related problems like inflation, poverty, and inequality cannot be fixed once and for all, there are many ways to help America's people and businesses. Unionizing to increase collective bargaining coverage, establishing a framework to setting the minimum wage, and taxing less are all ways to do so. Employers and employees shouldn't fight over pieces of a cake like in a zero-sum game. They can work together to expand the cake so everyone can get a larger piece.

\section*{References}

\begin{enumerate}
\item M. Bruenig, ``McDonald's Workers in Denmark Won Good Pay and Benefits Through Striking,'' 27 September 2021. [Online]. Available: \url{https://jacobin.com/2021/09/denmark-mcdonalds-labor-unions-strikes-wages-benefits}
\item E. H. Dyvik, ``Global price of a Big Mac as of July 2024, by country,'' 2025.
\item National Employment Law Project, ``Big Business, Corporate Profits, and the Minimum Wage,'' July 2012. [Online]. Available: \url{https://www.nelp.org/app/uploads/2015/03/NELP-Big-Business-Corporate-Profits-Minimum-Wage.pdf}
\item U.S. Bureau of Labor Statistics, ``Characteristics of minimum wage workers, 2022,'' August 2023. [Online]. Available: \url{https://www.bls.gov/opub/reports/minimum-wage/2022/#:~:text=Among%20major%20occupational%20groups%2C%20service,preparation%20and%20serving%2Drelated%20jobs}
\item Statista Research Department, \url{https://www.statista.com/statistics/681809/global-restaurant-industry-market-size-by-country/}, 2016.
\item U.S. Bureau of Labor Statistics, ``CPI Inflation Calculator,'' Washington, D.C., 2025.
\item Euractiv, ``The EU's minimum wage directive explained,'' YouTube, 2022.
\item European Commission, ``Adequate minimum wages for workers in the EU,'' 15 November 2024. [Online]. Available: \url{https://employment-social-affairs.ec.europa.eu/news/adequate-minimum-wages-workers-eu-2024-11-15_en#:~:text=Statutory%20minimum%20wages%20will%20need,with%20an%20automatic%20indexation%20mechanism}
\item E. H. Dyvik, ``Global price of a Big Mac as of July 2024, by country,'' 2024.
\item U.S. Department of Labor, ``UNION MEMBERS-2024,'' 2025.
\item Drexel University, ``Minimum Wage is Not Enough: A True Living Wage is Necessary to Reduce Poverty and Improve Health,'' December 2021. [Online]. Available: \url{https://drexel.edu/hunger-free-center/research/briefs-and-reports/minimum-wage-is-not-enough/}
\item National Conference of State Legislatures, 1 January 2025. [Online]. Available: \url{https://www.ncsl.org/labor-and-employment/state-minimum-wages#:~:text=Currently%2C%2034%20states%2C%20territories%20and,wage%20below%20%247.25%20per%20hour}
\item E. H. Dyvik, ``Global price of a Big Mac as of July 2024, by country,'' 2024.
\end{enumerate}

\end{document}
