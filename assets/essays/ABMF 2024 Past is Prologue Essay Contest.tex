\documentclass{article}
\usepackage[utf8]{inputenc}
\usepackage{geometry}
\usepackage{setspace}
\usepackage{hyperref}

% Set line spacing to 1.5 and add space between paragraphs to simulate "double" paragraph spacing
\onehalfspacing
\setlength{\parskip}{1.5em}

% Hyperlink setup for clickable citations and URLs
\hypersetup{
    colorlinks=true,
    linkcolor=blue,
    filecolor=magenta,      
    urlcolor=blue,
    citecolor=blue,
}

\begin{document}

\begin{center}
    \Large{\textbf{2024 Past is Prologue Essay Contest: Air Dominance that enabled the D-Day invasions}}
\end{center}

\vspace{1.5em}

"Let your plans be dark and impenetrable as night, and when you move, fall like a thunderbolt" \cite{1}.

This exactly describes how on June 6, 1944, troops from the U.S., U.K., Canada, Australia, Belgium, Czechoslovakia, France, Greece, the Netherlands, New Zealand, Norway, and Poland among others joined together and prepared for the largest amphibious invasion in history-D-Day \cite{2}.

On this day, it was the Air Force that shaped the Allied plans to be as "impenetrable as night" and to be like "thunderbolts."

Advanced technology such as airborne ground-scanning radar systems optimized navigation and allowed bombing through clouds and through the dark.

New equipment, including long-range bombers, naval aircraft carriers, and easily maneuverable fighters were manufactured.

In fact, air superiority depended on fast fighter planes so much that throughout WWII nations competed to build the fastest fighter plane.

The Allies seized air dominance through strategic bombing and tactical control of the battlefield by air.

Gaining control of airspace above the battlefield meant light bombers could support ground troops by strafing enemy targets.

In addition, bombing German war production and supply lines cut off German supplies of oil.

Mining Germany's coastal waters also prevented U-boats from operating against Allied shipping.

In Normandy, Allied air forces attacked the French railroad system, as well as critical bridges and tunnels.

Bombers simultaneously strangled German oil refineries and warplanes, which made the enemy incapable of reinforcing counterattacks.

Aerial reconnaissance planes, loaded with cameras, photographed beachheads and German positions.

This updated the Allies with information on the enemy's defenses. The Allies tricked the Germans into thinking that the invasion would take place at Pas-de-Calais instead of Normandy.

There, aircraft bombings and decoy dummy landing craft mimicked a large-scale invasion \cite{3}.

This strategy is putting Sun Tzu's words into action: "Hold out baits to entice the enemy. Feign disorder, and crush him" \cite{1}.

The Normandy Invasion began by dropping paratroopers and troop carrier units in the early morning.

Lightweight glider planes landed without detection from German defenses. Meanwhile, aerial minesweepers cleared water-submerged mines to clear a path for the invasion fleet from midnight till dawn.

Over 2,200 Allied bombers started attacking Normandy around midnight \cite{4}, hammering down beach defenses, communications systems, gun emplacements, and marshalling yards.

Fighter aircraft covered the ground troops in battle and flew beside bomber planes to protect them from the German Luftwaffe.

Air transport was also used to evacuate a third of wounded soldiers from the Normandy battlefields \cite{5}.

The Allied air assets above gave a "show of force" that could alter the ground battles.

The D-Day aircraft not only ensured military power, but they also uplifted the soldiers on the ground with the energy to charge into battle.

This is what German General Fritz Bayerlein remarked after the Cobra air attacks: "The aircraft flew over us continuously, passing above us like a conveyor belt... My front positions resembled a scene from the moon, and at least 70\% of my troops were out of action dead, buried, or stunned. All of my forward tanks were disabled, and the roads were practically impassable."

In order to support such a strong Air Force, the U.S. nearly tripled manufacturing and production on the home front from 1939 to 1944. The American economy shifted to wartime production to support the military needs of the Allies \cite{6}.

American companies produced 300,000 aircraft, 124,000 ships, over 100,000 tanks and armored vehicles and 2,400,000 trucks \cite{7}. The nation's GNP rose from \$99.7 billion in 1940 to \$212 billion in 1945 \cite{8}.

After the war, the U.S. held two thirds of global gold reserves, manufactured more than half of all products in the world, and covered one third of global exports \cite{9}.

The once steadily rising American dollar became the reserve currency for international commerce and trade.

WWII transformed the U.S. from an isolationist mid-level power to a global economic leader.

Major responsibilities also arose with this new global power, including the creation of the United Nations in 1945. The U.S. established new markets for American goods and spread capitalism by investing to rebuild the nations deteriorated by WWII \cite{9}.

During WWII, airpower was the leading global weaponry that each nation aimed to gain control over.

The success of the U.S. mastering air combat led to many more triumphs throughout the war.

Nowadays, the top pioneering advancements include space technology, cyberspace defense, and quantum technology.

The U.S. Space Force the newest military branch was established to consolidate the effectiveness of operations of other military branches.

The Space Force joins forces on land, air, and space, and helps coordinate national defense and global attacks.

The U.S. Cyber Command was found to fight in the domain of cyberspace.

In the Information Age that we live in today, we have to recognize the necessity of computer security, the protection of information systems, and the prevention of equipment malfunctions, even up to the national level.

Effective remote attacks can prevent physical battle and injuries. Quantum mechanics is another breakthrough that scientists have figured out.

When harnessed to enhance computing, sensing, and communication, quantum mechanics can revolutionize military battles and national defense with its impenetrable power and speed.

The past feats from our previous generations prove that we Americans can achieve great things.

In order for us, the future leaders of the nation, to summit the same heights as the G.I. generation, we have to keep up with this fast-paced world, and we can never let our guards down.

This world does not need to get stricken with warfare and bloodshed. In the end, "peace" is the most true and triumphant thing of all. "... to fight and conquer in all your battles is not supreme excellence; supreme excellence consists in breaking the enemy's resistance without fighting" \cite{1}.

\section*{Research materials}
\begin{thebibliography}{99}

\bibitem{1} S. Tzu, The Art of War.

\bibitem{2} "10 things you might not know about D-Day," Royal British Legion, [Online]. Available: \url{https://www.britishlegion.org.uk/stories/ten-things-you-might-not-know-about-d-day}. [Accessed 2024 March 28].

\bibitem{3} "D-Day's Parachuting Dummies and Inflatable Tanks," Imperial War Museums, 2024. [Online]. Available: \url{https://www.iwm.org.uk/history/d-days-parachuting-dummies-and-inflatable-tanks}. [Accessed 26 March 2024].

\bibitem{4} "Normandy landings," Wikepedia, 6 March 2024. [Online]. Available: \url{https://en.wikipedia.org/wiki/Normandy_landings}. [Accessed 28 March 2024].

\bibitem{5} "OPERATION OVERLORD: D-Day," National Museum of the United States Air Force, [Online]. Available: \url{https://www.nationalmuseum.af.mil/visit/Museum-Exhibits/Fact-Sheets/Display/Article/1789416/operation-overlord-d-day}. [Accessed 28 March 2024].

\bibitem{6} U.S. Department of Defense, "During WWII, Industries Transitioned From Peacetime to Wartime Production," U.S. Department of Defense, 27 March 2020. [Online]. Available: \url{https://www.defense.gov/News/Feature-Stories/story/article/2128446/during-wwii-industries-transitioned-from-peacetime-to-wartime-production/}. [Accessed 28 March 2024].

\bibitem{7} "PRODUCING THE WEAPONS OF WAR," Western CT State University Archives' Digital Collections, [Online]. Available: \url{https://archives.library.wcsu.edu/omeka/exhibits/show/world-war-ii-in-life-magazine-/weapons-of-war}. [Accessed 28 March 2024].

\bibitem{8} "Take A Closer Look: America Goes to War," The National Museum of New Orleans, [Online]. Available: \url{https://www.nationalww2museum.org/students-teachers/student-resources/research-starters/america-goes-war-take-closer-look}. [Accessed 28 March 2024].

\bibitem{9} "Great Responsibilities and New Global Power," The National WWII Museum of New Orleans, 23 October 2020. [Online]. Available: \url{https://www.nationalww2museum.org/war/articles/new-global-power-after-world-war-ii-1945}. [Accessed 28 March 2024].

\bibitem{10} "U.S. Commercial Shipbuilding in a Global Context," Congressional Research Service, 15 November 2023. [Online]. Available: \url{https://crsreports.congress.gov/product/pdf/IF/IF12534}. [Accessed 27 March 2024].

\bibitem{11} "The U.S. Dollar as the World's Dominant Reserve Currency," 15 September 2022. [Online]. Available: \url{https://crsreports.congress.gov/product/pdf/IF/IF11707}. [Accessed 28 March 2024].

\bibitem{12} N. H. a. H. Command, "US Ship Force Levels," Naval History and Heritage Command, 17 November 2017. [Online]. Available: \url{https://www.history.navy.mil/research/histories/ship-histories/us-ship-force-levels.html}. [Accessed 27 March 2024].

\bibitem{13} M. C. Everett, "Quantum Technology," 20 July 2023. [Online]. Available: \url{https://www.airuniversity.af.edu/Wild-Blue-Yonder/Articles/Article-Display/Article/3457421/quantum-technology/}. [Accessed 28 March 2024].

\bibitem{14} U. N. Institute, "Can We Modernize U.S. Shipbuilding?," U.S. Naval Institute, January 1966. [Online]. Available: \url{https://www.usni.org/magazines/proceedings/1966/january/can-we-modernize-us-shipbuilding}. [Accessed 27 March 2024].

\bibitem{15} S. \&. E. Indicators, "The State of U.S. Science and Engineering 2022," Science \& Engineering Indicators, 2022. [Online]. Available: \url{https://ncses.nsf.gov/pubs/nsb20221/u-s-and-global-research-and-development}. [Accessed 27 March 2024].

\end{thebibliography}

\end{document}